% -- TODO --
% This chapter isn't done yet:
% look into removing files more

\chapter{Tracking Path Changes}
\section{Tracking Path Changes}

\subsection{Removing files locally}
To remove a file from Git, you have to remove it from your tracked files (more accurately, remove it from your staging area) and then commit. The \gitinline{git rm} command does that, and also removes the file from your working directory so you don’t see it as an untracked file the next time around.\newline
This is different than just doing \gitinline{rm <file>}, because then git will just mark the file as “Changes not staged for commit”

\subsection{Removing files from remote}

\subsection{Remove tracking but keep locally}

\begin{gitBashBox}
rm --cached <file>    
\end{gitBashBox}
Keeps the file in your working tree but removes it from your staging area.

\subsection{Batch removal}
\begin{gitBashBox}
rm log/\*.log
\end{gitBashBox}\footnotemark
Removes all log files in the log/ directory.



\footnotetext{we need to escape the * character}

\subsection{Renaming Files}
Unlike many other VCSs, Git doesn’t explicitly track file movement.

\noindent If you want to rename a file you can do:
\begin{gitBashBox}
mv <file_from> <file_to>
\end{gitBashBox}

This works the same way as the Linux command  \gitinline{mv dir1 dir2} 
In this scenario if the dir2 directory exists, the command will move dir1 inside dir2. If dir2 doesn’t exist, dir1 will be renamed to dir2.\newline
The same happens with \gitinline{git mv dir1 dir2}, the only difference is that the changes are staged automatically.