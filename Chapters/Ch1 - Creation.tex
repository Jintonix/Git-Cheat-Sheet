\chapter{Creation}
\chapteroverlay
\section{How to create a project using Git}
Create a project locally. Then run \gitinline{git init} -- which creates a repository in the folder you are currently in.\\
Alternativley you can use \gitinline{git init <name>}  --which basically does \gitinline{mkdir <name>} and \gitinline{git init <name>} in one line and creates a folder in the directory you are in.\\

\subsection{Connect to a Remote Repository}
Once you have a git project set up locally, you will want to connect it to a remote host. Use the Git hosting service of your choosing and create a project. Then use the \textit{url} (either SSH or HTTP) and connect it to your local.
\begin{gitBashBox}
remote add origin <url>
branch -M main
push -u origin main
\end{gitBashBox}

\subsection{Changing upstream}
\textbf{\textcolor{red}{WORK IN PROGRESS}}

\subsection{How to clone a repository}
Sometimes you want to work on an already existing project. 
\begin{gitBashBox}
clone <url>
\end{gitBashBox}
This creates a repository in a new dir in your cwd named after the project in your Git hosting service.\\
\noindent\gitinline{git clone <url> myFolder}
Does the same, but replaces the project name with a specific one you set.\\
\noindent\gitinline{git clone <url> .}
Creates the repository in your cwd.