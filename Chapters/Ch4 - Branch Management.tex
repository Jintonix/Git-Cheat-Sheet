\chapter{Branch Management}
\chapteroverlay
\chapterunderlay

\section{Branching}
To create one locally you use the command:\\
\$git checkout -b my-feature\\
This combines both:\\
\$git branch my-feature → creates the branch\\
\$git checkout my-feature → switches to it\\

After that it needs to be pushed to the remote via:\\
\$git push -u origin my-feature\\

\section{Branch vs. Tag}
Differences between Git tags and branches

Tags and branches are both used for version control of your code base. Their functions complement each other, and they are designed to be used together.

A branch is often used for new features and fixing bugs. It allows you to work without impacting the main code base. Once your work is complete, you can incorporate your changes into the application by merging the branch back into the main code base. This allows multiple people to work on different aspects of the project simultaneously. It also provides a way to experiment with new ideas without risking the main code base's stability.

Unlike branches, tags are not intended for ongoing development. Tags mark a specific point in the repository’s history to give developers an easy way to reference important milestones in the development timeline.
When to use branches

Imagine that you want to add a new feature to a software project’s code base but you aren’t sure if the new feature will work as expected. You want to experiment without affecting the main code base.

In this scenario, use a Git branch to create a separate line of development for the new feature. This avoids affecting the main code base. Once the feature is complete and tested, you can merge the branch back into the main code base. Depending on your organization’s process, you may choose to delete the branch to prevent clutter.
When to use tags

Now, suppose that you complete and test the new feature. You want to release the new software version to the users. In this case, use a Git tag to mark the current state of the code base as a new version release.

You can name the tag to reflect the version number (such as v1.2.3) and include a brief description of the changes in the release. This allows you to easily reference the specific version of the released code base. It is also easy to roll back to the previous version if necessary.
Using 

\subsection{Taking changes to new branch}
Every now and then you may be working on a branch and have an idea. Start working on that idea and then realize that perhaps this idea should be its own branch. When this happens you can move your worked on files into a new branch. Here it's important to distinguish between 
\begin{enumerate}
    \item changes have been committed
    \item changes are unstaged
\end{enumerate}

\subsubsection{No commits}
1. Stash your changes: \footnotemark
\begin{gitBashBox}
git stash push -m "work in progress"
\end{gitBashBox}\newline
2. Switch to the branch you want to branch off from: \gitinline{git switch <branch>}\newline
3. Create a new branch and switch to it \gitinline{git switch -c <new_branch>}\newline
4. Apply your stashed work: \gitinline{git stash pop}

\footnotetext{\gitinline{git stash} is just a shorthand for \gitinline{git stash push}. It does however allow us to use flags like \gitinline{-m}}

\subsubsection{Previous Commits}
1. Stay on the wrong branch, note the last commit hash: \gitinline{git log}

2. Switch to main, create the new branch:
\begin{gitBashBox}
    git switch <branch>
    git switch -c <new_branch>
\end{gitBashBox}

3. Cherry-pick your commits:
\begin{gitBashBox}
    git cherry-pick <commit-hash>
\end{gitBashBox}

